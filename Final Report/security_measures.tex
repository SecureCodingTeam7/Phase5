\chapter{Security Measures}

\section{PDF Password Protection}
The TAN List of our customers are password protected. That ensures that only customer himself can access the transaction codes. To porotect the pdfs we used thethird party library FDPI.
\section{HTTPS}
Our webservice uses the HTTPS protocoll to encrypt all transmitted data to and from our clients.
This prevents attackers from reading sensitive data via man-in-the-middle attacks.
\section{Strong password policy}
We enforce the user to choose a string password to minimize the possible threat based on brute force attacks.
\section{Secure Session Management}
The session cookie of an user is only transmitted via HTTP, can not be accessed via javascript and expires after 30 minutes. So attackers cannot bypassing the authentication schema by stealing cookies of logged in users.
\section{Prepared Statements for SQL database}
We protected our web service against sql injections by using prepared statements for database requests.
\section{CSRF Token}
\section{Secure Generation of Transaction Codes}
\section{Protection of C Parser against Heap/Buffer Overflows}
\section{Secure Handling of uploaded files}
Uploaded Files for batch transaction are handled in a secure manner to avoid damage of uploaded malicious code. This includes the usage of the default PHP temporary folder. All files are stored in /tmp and get a randomly chosen name. This has two effects. First of all files are not stored at the webroot and can not be access via an unsecure webserver. Secondly, renaming files before passing them to the c parser, prevents malicous code in the filename to be executed.
\newline
Another precoution ew took, is to delete this files right after parsing them, so that malicous code can't be executed later on. 


