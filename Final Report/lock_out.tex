\chapter{Lock out Mechanism}

To avoid brute forcing attacks when logging in and transferring money (TANs) we implemented a mechanism which blocks the user after 5 attempts with invalid input. Thus we created a new database field in the \textit{users} table (lock\_counter, int). 

\section{Affected Files}
\begin{itemize}
	\item c\_user.php : line 898 new methods
	\item c\_user.php : lines 455, 469, 877  method call resetLockCounter()
		\item c\_user.php : lines 459, 472, 883  method call incrementLockCounter()
\end{itemize}

\section{Additions}

\begin{lstlisting}[caption = New methods]
function incrementLockCounter() {
	$connection = new PDO( DB_NAME, DB_USER, DB_PASS );
	$sql = "SELECT lock_counter FROM users WHERE email = :email LIMIT 1";
	$stmt = $connection->prepare( $sql );
	$stmt->bindValue( "email", $this->email, PDO::PARAM_STR );
	$stmt->execute();
	$result = $stmt->fetch();
	if ($result['lock_counter'] < 5) {
		$sql = "update users set lock_counter = lock_counter + 1 where email = :email";
		$stmt = $connection->prepare( $sql );
		$stmt->bindValue( "email", $this->email, PDO::PARAM_STR );
		$stmt->execute();
		return false;
	} else {
		$sql = "update users set is_active = 0, lock_counter = 0 where email = :email";
		$stmt = $connection->prepare( $sql );
		$stmt->bindValue( "email", $this->email, PDO::PARAM_STR );
		$stmt->execute();
		return true;
	}
	
	$connection = null;
}
function resetLockCounter() {
	$connection = new PDO( DB_NAME, DB_USER, DB_PASS );
	$sql = "update users set lock_counter = 0 where email = :email";
	$stmt = $connection->prepare( $sql );
	$stmt->bindValue( "email", $this->email, PDO::PARAM_STR );
	$stmt->execute();
	$connection = null;
}
\end{lstlisting}